\documentclass[12pt,a4paper,landscape]{article}
\usepackage[utf8]{inputenc}
\usepackage{amsmath}
\usepackage{amsfonts}
\usepackage{amssymb}
\usepackage{graphicx}
\usepackage{multirow}
\usepackage{longtable}
\usepackage[left=0.3cm,right=0.3cm,top=2cm,bottom=2cm]{geometry}
\usepackage{array}
\usepackage{enumitem}
\begin{document}
\begin{center}
% Définition du titre
\begin{LARGE}
IFNTI : Maquette \VAR{titre}
\end{LARGE}
\end{center}
\begin{center}
% Définition du tableau
\setlength{\tabcolsep}{0pt}
\renewcommand{\arraystretch}{1}

% Définition d'une nouvelle liste personnalisée avec un espacement nul
\newlist{myitemize}{itemize}{1}
\setlist[myitemize,1]{label=--,left=0pt,labelsep=0pt,topsep=0pt,partopsep=0pt,parsep=0pt,itemsep=0pt}


\begin{longtable}{|m{2cm}|m{6cm}|m{3cm}|m{1.5cm}|m{2cm}|m{3cm}|m{3cm}|}
\hline
Semestre & Intitulé UE & Type UE & Crédit UE & Volume Horaire (H) & \centering Enseignant principal & \centering Enseignant secondaire \arraybackslash \\ 
\hline

\BLOCK{ for semestre_data in semestres_data }

\hline
\multirow{\VAR{semestre_data.ues|length}}{*}{\VAR{semestre_data.semsetre}} 
\BLOCK{ for ue in semestre_data.ues}
& \VAR{ue.libelle} & \VAR{ue.type} & \VAR{ue.nbreCredit} & \VAR{ue.heurs} & \VAR{ue.enseignant} &  \\
\cline{2-7} 
\BLOCK{ endfor }

\BLOCK{ endfor }

\cline{1-1} \cline{6-6}
Total &  &  & 64 & 8 &  &  \\ 
\hline 
\end{longtable}

\end{center}



\end{document}

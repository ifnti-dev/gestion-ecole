\documentclass[12pt,a4paper,landscape]{article}
\usepackage[utf8]{inputenc}
\usepackage{amsmath}
\usepackage{amsfonts}
\usepackage{amssymb}
\usepackage{graphicx}
\usepackage{multirow}
\usepackage{longtable}
\usepackage[left=0.3cm,right=0.3cm,top=2cm,bottom=2cm]{geometry}
\usepackage{array}
\usepackage{enumitem}
\begin{document}
\begin{center}
% Définition du titre
\begin{LARGE}
IFNTI : Maquette \VAR{titre}
\end{LARGE}
\end{center}
\begin{center}
% Définition du tableau
\setlength{\tabcolsep}{0pt}
\renewcommand{\arraystretch}{1}

% Définition d'une nouvelle liste personnalisée avec un espacement nul
\newlist{myitemize}{itemize}{1}
\setlist[myitemize,1]{label=--,left=0pt,labelsep=0pt,topsep=0pt,partopsep=0pt,parsep=0pt,itemsep=0pt}

\begin{longtable}{|m{2cm}|m{6cm}|m{3cm}|m{7cm}|m{1.5cm}|m{2cm}|m{3cm}|m{3cm}|}
\hline
\centering Semestre & \centering Intitulé UE & \centering Type UE & \centering Crédit UE & \centering Volume Horaire (H) & \centering Enseignant Responsable & \centering Enseignant principal \arraybackslash \\ 
\hline
\BLOCK{ for semestre_data in semestres_data }
\multirow{\VAR{semestre_data.ues_length}}{*}{\VAR{semestre_data.semsetre.libelle}} 
\BLOCK{ for ue in semestre_data.ues }
 & \centering \VAR{ue.libelle} 
 & \centering \VAR{ue.type}  
 
 &  
	\centering \VAR{ue.nbreCredits} 
 & 
 	\centering \VAR{ue.heures}  
 & 
   \BLOCK{ if ue.matiere_set.all()  }
 	\begin{myitemize}
	\BLOCK{ for matiere in ue.matiere_set.all() }
	\item[] \centering \VAR{matiere.enseignant}
	\item[] \hrulefill
	\BLOCK{ endfor }
	\end{myitemize}
	\BLOCK{ endif }
 & \centering \VAR{ue.enseignant} \arraybackslash  \\
\cline{2-4} \cline{6-8} 
\BLOCK{ endfor }
\cline{1-1} \cline{5-5}
\BLOCK{ endfor }
Total &  &  &  & 64 & 8 &  &  \\ 
\hline 
\end{longtable}

\end{center}



\end{document}

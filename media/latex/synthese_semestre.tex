\documentclass[10pt,a4paper,landscape]{article}
\usepackage[utf8]{inputenc}
\usepackage[T1]{fontenc}
\usepackage{amsmath}
\usepackage{amsfonts}
\usepackage{amssymb}
\usepackage{graphicx}
\usepackage{lscape}
\usepackage{multirow}
\usepackage{array}
\usepackage[OT1]{fontenc}
\usepackage[utf8]{inputenc}
\usepackage[none]{hyphenat}
\usepackage{fancyhdr}
\usepackage{eso-pic, graphicx}
\usepackage{geometry}[left=0.5cm, bottom=0.5cm, right=0.5, top=0.5cm]
\usepackage[french]{babel}



\graphicspath{ {\VAR{image_path}} }
\newcommand\BackgroundPic{
\put(-4,0){
\parbox[b][\paperheight]{\paperwidth}{
\includegraphics[width=\paperwidth,height=\paperheight]{CharteGraphiquePresentationPageDeGarde}
}}}



\begin{document}  

 
\begin{landscape}





%faire une boucle qui compte et additionne pour chaque ue , le nombre de matiere , recuperer la valeur total de matiere et y ajouter 2
%la valeur ainsi recuperer , on fait une boucle pour creer le begin tabular et  la valeur correspond au nombre de |c|


\rotatebox{270}{
\vspace{-8cm}
\hspace{-3cm}
\begin{tabular}{\VAR{colonnes}}
%ensuite pour le cline on par de 3 jusqua cette valeur recuperer
\cline{3-\VAR{nbre_colonnes}}
%ensuite ici on met la premierer ligne (\multicolumn{2}{c|}{} &) et  on fait une boucle pour recuperer les ue , pour chaque ue , on recupere une valeur correspondante au nombre de matiere que comprend l'ue et on la met ensuite dabs le multicolumn , et dans les crochet , de fin , on ajoute le libelle de l'ue
\multicolumn{2}{c|}{} & 
\BLOCK{for ue in range(nbre_ues -1)}
  \multicolumn{\VAR{data_releve[ue]['nbre_matieres']}}{p{3cm}|}{\centering \VAR{data_releve[ue]['ue'].libelle}} & 
\BLOCK{endfor}
\multicolumn{\VAR{data_releve[nbre_ues -1]['nbre_matieres']}}{p{3cm}|}{\centering \VAR{data_releve[nbre_ues -1]['ue'].libelle}}\\ 

% on vient ensuite ici apres le hline , on met (Prénom & Nom & ) et on boucle ensuite pour recuperer dans l'odre des ue , les libellle de chacune des matieres des ue et on les met dans les crochet avec la proprieter centring
\hline 
Nom & Prenom & 
\BLOCK{for ue in range(nbre_ues -1)}
  \BLOCK{for matiere in data_releve[ue]['matieres_ue']}
    \rotatebox{90}{\parbox[c]{2.5cm}{\centering \VAR{matiere['matiere'].libelle} }}  & 
  \BLOCK{endfor} 
\rotatebox{90}{\parbox[c]{2.5cm}{\centering UE }} & 
\BLOCK{endfor} 

\BLOCK{for matiere in data_releve[nbre_ues -1]['matieres_ue']}
\rotatebox{90}{\parbox[c]{2.5cm}{\centering \VAR{matiere['matiere'].libelle} }}  & 
\BLOCK{endfor} 
\rotatebox{90}{\parbox[c]{2.5cm}{\centering UE }} \\

% ensuite ici on ferme en mettant un intervale de la valeur recuperer , qui par de cet valeur a cette valeur dans l'element cline
\cline{\VAR{nbre_colonnes}-\VAR{nbre_colonnes}}

%ici on fait une boucle pour compter le nombre d'eleve et pour chaque eleve , on genere une ligne de l'element hline qui prend le prenom , le nom , et la moyenne pour chaque matiere dans l'ordre 
\hline 
\VAR{etudiant.prenom} & \VAR{etudiant.nom} & 
\BLOCK{for ue in range(nbre_ues -1)}
  \BLOCK{for matiere in data_releve[ue]['matieres_ue']}
    \VAR{matiere['moyenne_matiere']}  & 
  \BLOCK{endfor} 
  \VAR{data_releve[ue]['moyenne']}  & 
\BLOCK{endfor} 

\BLOCK{for matiere in data_releve[nbre_ues -1]['matieres_ue']}
  \VAR{matiere['moyenne_matiere']}  & 
\BLOCK{endfor} 
\VAR{data_releve[nbre_ues -1]['moyenne']}  \\

\hline
\end{tabular} 

}

\end{landscape}

% un dictionnaire qui qui a le nombre de matiere par ue et le libelle de cette ue
% un tableau des libelle de chacune des matiere
% une donnee longueur tableau qui est le nombre de matiere total pour les ue + 2 
% j'ai besoin d'un dictionnaire qui pour chaque etudiant , regroupe son prenom , son nom et chacune des moyennes de chacune de ses matieres




\end{document}